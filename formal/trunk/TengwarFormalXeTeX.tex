%&program=xelatex
%&encoding=UTF-8 Unicode

\documentclass[11pt,a4paper]{article}
\usepackage{fontspec,hyperref,xltxtra}

\font\formalGR="Tengwar Formal Unicode/GR"
\font\formalAAT="Tengwar Formal Unicode/AAT"

\font\formalGRdescw="Tengwar Formal Unicode/GR:204=1"
\font\formalAATdescw="Tengwar Formal Unicode/AAT:Character Alternatives=Descending W-Tengwa;Sa-Rince Style=Swash"

\font\formalGRossefinal="Tengwar Formal Unicode/GR:207=1"
\font\formalAATossefinal="Tengwar Formal Unicode/AAT:Osse Hook=Word Final"

\font\formalGRossealways="Tengwar Formal Unicode/GR:207=2"
\font\formalAATossealways="Tengwar Formal Unicode/AAT:Osse Hook=Always"

\font\formalGRtilde="Tengwar Formal Unicode/GR:209=1"
\font\formalAATtilde="Tengwar Formal Unicode/AAT:Character Alternatives=Tilde Bar"

\font\formalGRcircumflex="Tengwar Formal Unicode/GR:210=1"
\font\formalAATcircumflex="Tengwar Formal Unicode/AAT:Character Alternatives=Circumflex A-Tehta"

\font\formalGRswash="Tengwar Formal Unicode/GR:211=1"
\font\formalAATswash="Tengwar Formal Unicode/AAT:Character Alternatives=Swash Sa-Rince"

\font\formalGRcarrierdot="Tengwar Formal Unicode/GR:213=1"
\font\formalAATcarrierdot="Tengwar Formal Unicode/AAT:Carrier Diacritic (for 'Full Writing')=Dot"

\font\formalGRcarrieracute="Tengwar Formal Unicode/GR:213=2"
\font\formalAATcarrieracute="Tengwar Formal Unicode/AAT:Carrier Diacritic (for 'Full Writing')=Acute"

\font\formalGRosseambiguous="Tengwar Formal Unicode/GR:214=1"
\font\formalAATosseambiguous="Tengwar Formal Unicode/AAT:Osse Diacritic (for 'Full Writing')=Dot on Ambiguous Osse"

\font\formalGRosseevery="Tengwar Formal Unicode/GR:214=2"
\font\formalAATosseevery="Tengwar Formal Unicode/AAT:Osse Diacritic (for 'Full Writing')=Dot on Every Osse"

\font\myformal="Tengwar Formal Unicode/GR:204=1;207=2"

\setmonofont{FreeMonoTengwar}



\begin{document}



\tableofcontents

\section{Tengwar Formal Unicode/AAT: Graphite and AAT options}



Here are samples of the different options. The options are specified by adding something to the font name in the \texttt{\char`\\font} command. The extended font names are all given in this document. A font specification will look like this:

\paragraph{} \texttt{\char`\\font\char`\\bla="font(/GR):feature=value(;feature=value)"}

\paragraph{} In this specification, the part \texttt{bla} is a string of your choice. On Mac OS X, the part \texttt{/GR} is optional. It tells \XeTeX{} to use the Graphite information. If you leave it away, it will use the AAT information, which only works on Mac OS X. Note that you can define as many features as you like:

\paragraph{} \texttt{\char`\\font\char`\\myformal="Tengwar Formal Unicode/GR:204=1;207=2"}

\paragraph{} After this \texttt{\char`\\font} command has been defined, you can use \texttt{\char`\\myformal} as a command, for instance as follows:

\paragraph{} \texttt{\char`\\myformal   \char`\\normalfont}

\paragraph{} This will look like this: \myformal   \normalfont



\subsection{Character Alternatives}



\subsubsection{Descending W-Tengwa}


Default

\formalGR  

\noindent \texttt{"Tengwar Formal Unicode/GR:204=1"}

\formalGRdescw  

\noindent \texttt{"Tengwar Formal Unicode/AAT:Character Alternatives=Descending W-Tengwa;"}

\formalAATdescw   \normalfont



\subsubsection{Tilde Bar}


Default

\formalGR 

\noindent\texttt{"Tengwar Formal Unicode/GR:209=1"}

\formalGRtilde 

\noindent\texttt{"Tengwar Formal Unicode/AAT:Character Alternatives=Tilde Bar"}

\formalAATtilde  \normalfont



\subsubsection{Circumflex A-Tehta}


Default

\formalGR 

\noindent\texttt{"Tengwar Formal Unicode/GR:210=1"}

\formalGRcircumflex 

\noindent \texttt{"Tengwar Formal Unicode/AAT:Character Alternatives=Circumflex A-Tehta"}

\formalAATcircumflex  \normalfont



\subsubsection{Swash Sa-Rince}


Default

\formalGR 

\noindent\texttt{"Tengwar Formal Unicode/GR:211=1"}

\formalGRswash 

\noindent \texttt{"Tengwar Formal Unicode/AAT:Character Alternatives=Swash Sa-Rince"}

\formalAATswash  \normalfont



\subsection{Osse Hook}


Default

\formalGR  \normalfont



\subsubsection*{Word Final}


\texttt{"Tengwar Formal Unicode/GR:207=1"}

\formalGRossefinal 

\noindent \texttt{"Tengwar Formal Unicode/AAT:Osse Hook=Word Final"}

\formalAATossefinal 



\subsubsection*{Always}


\texttt{"Tengwar Formal Unicode/GR:207=2"}

\formalGRossealways 

\noindent \texttt{"Tengwar Formal Unicode/AAT:Osse Hook=Always"}

\formalAATossealways  \normalfont



\subsection{Carrier Diacritic (for ‘Full Writing’)}


Default

\formalGR  



\subsubsection*{Dot}


\texttt{"Tengwar Formal Unicode/GR:213=1"}

\formalGRcarrierdot  

\noindent \texttt{"Tengwar Formal Unicode/AAT:Carrier Diacritic (for 'Full Writing')=Dot"}

\formalAATcarrierdot   



\subsubsection*{Acute}


\texttt{"Tengwar Formal Unicode/GR:213=2"}

\formalGRcarrieracute  

\noindent \texttt{"Tengwar Formal Unicode/AAT:Carrier Diacritic (for 'Full Writing')=Acute"}

\formalAATcarrieracute   \normalfont



\subsection{Osse Diacritic (for ‘Full Writing’)}


Default

\formalGR  



\subsubsection*{Dot on Ambiguous Osse}


\texttt{"Tengwar Formal Unicode/GR:214=1"}

\formalGRosseambiguous  

\noindent \texttt{"Tengwar Formal Unicode/AAT:Osse Diacritic (for 'Full Writing')=Dot on Ambiguous Osse"}

\formalAATosseambiguous  



\subsubsection*{Dot on every osse}


\texttt{"Tengwar Formal Unicode/GR:214=2"}

\formalGRosseevery  

\noindent \texttt{"Tengwar Formal Unicode/AAT:Osse Diacritic (for 'Full Writing')=Dot on Every Osse"}

\formalAATosseevery  



\newpage



\section{Sample Text}


\subsection{Graphite}
\formalGR
  ‍   ⸱‍  ‍ ⸱ \\
    ‍....  ‍  \\
  ‍   ‍ ‍ ⁊ ‍  \\
...  ....   \\
         ‍ \\
 ‍ ‍‍   ⁊  ‍ ‍ \\
 ‍ ⁊  . . . .  ‍ ‍‍   \\
   ‍ . . . . . ‍ ‍   \\
      ⁊    \\
   ‍⸱‍   ‍  \\
     ‍⸱ ⁊  ‍ \\
    ⸱    \\
⸱  ‍  .... ‍‍    \\

\subsection{AAT}
\formalAAT
  ‍   ⸱‍  ‍ ⸱ \\
    ‍....  ‍  \\
  ‍   ‍ ‍ ⁊ ‍  \\
...  ....   \\
         ‍ \\
 ‍ ‍‍   ⁊  ‍ ‍ \\
 ‍ ⁊  . . . .  ‍ ‍‍   \\
   ‍ . . . . . ‍ ‍   \\
      ⁊    \\
   ‍⸱‍   ‍  \\
     ‍⸱ ⁊  ‍ \\
    ⸱    \\
⸱  ‍  .... ‍‍    \\

\end{document}
